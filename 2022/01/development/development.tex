\documentclass[10pt]{article}
\usepackage[T1]{fontenc}
\usepackage{amsmath, amsfonts, mathtools}


\begin{document}


L'equazione nonlineare di Schrödinger (NLSE) compare nei più disparati campi della Fisica, come l'ottica nonlineare, la materia condensata, la fisica dei plasmi. In tutti questi scenari, l'equazione descrive la dinamica di inviluppi in presenza di dispersione e nonlinearità debole. Quindi una domanda che viene spontanea è quella di capire se sia possibile derivare questa equazione senza ricorrere ad una particolare situazione fisica. Ciò è particolarmente illuminante dal punto di vista matematico, oltre ad essere un utile esercizio. In questo post esploreremo un metodo che ci permette di derivare l'equazione in generale, a partire da una generica equazione alle derivate parziali, usando solamente assunzioni fondamentali sulle sue proprietà matematiche delle soluzioni. 
Sia in generale un'equazione alle derivate parziali in $d$ dimensioni per un campo scalare $u$ espressa come 
\begin{equation}
	L(\partial_t, \nabla)u = G(u),
\end{equation}
dove $L$ è un operatore lineare, e $G$ una generica funzione, nonlineare di $u$ e le sue derivate.
Per soluzioni dall'ampiezza ridotta, il termine nonlineare viene assunto essere trascurabile, perciò le soluzioni saranno
\begin{equation}\label{solution}
	u = \epsilon\psi \exp[i(\mathbf{k} \cdot \mathbf{r} - \omega t)],
\end{equation}
con $\omega$ e $\mathbf{k}$ che soddisfano la relazione di dispersione
\begin{equation}
	L(-i\omega, i\mathbf{k}) = 0.
\end{equation}
in generale, questa relazione può ammettere diverse soluzioni $\omega$ per uno stesso $\mathbf{k}$. Sia $\omega(\mathbf{k})$ una di queste.
Prendendo la soluzione espressa in Eq. (\ref{solution}), la relazione di dispersione può essere scritta come 
\begin{equation}\label{dispersion}
	[i\partial_t - \omega(i\partial_{\mathbf{r}})] \ \epsilon\psi\exp[i(\mathbf{k} \cdot \mathbf{r} - \omega t)],
\end{equation}
promuovendo $\omega$ ad un operatore.
Ora introduciamo una correzione: nel caso di nonlinearità piccole, la pulsazione dipenderà anche da $\epsilon^2 |\psi|^2$. Quindi potremo sostituire $\omega(\mathbf{k})$ con $\Omega(\mathbf{k}, \epsilon^2|\psi|^2)$. Si noti $\Omega(\mathbf{k}, 0)=\omega(\mathbf{k})$.
Inoltre, l'ampiezza $\psi$ diventa dipendente dalle variabili ``lente'' $T=\epsilon t$ e $\mathbf{X} = \epsilon \mathbf{x}$. Possiamo sostituire formalmente $\partial_t \longrightarrow \partial_t + \epsilon\partial_T$ e $\partial_{\mathbf{x}} \longrightarrow \partial_{x}+\epsilon \nabla$, dove $\nabla$ è il gradiente rispetto alla variabile ``lenta''. 
La relazione Eq. (\ref{dispersion}) diventa
\begin{equation}
	[i\omega + \epsilon \partial_T - \Omega(\mathbf{k}-i\epsilon\nabla, \epsilon^2|\psi|^2)]\psi = 0.
\end{equation}
Dato che $\epsilon<<1$, possiamo espandere in serie questa equazione attorno a $\epsilon=0$, ottenendo
\begin{equation}
	i(\partial_T + \mathbf{v}_g \cdot \nabla)\psi + \epsilon[\nabla \cdot (D\nabla\psi) + \gamma|\psi|^2\psi]=0
\end{equation}
dove $\mathbf{v}_g=\nabla_\mathbf{k}\omega$ è la velocità di gruppo, e $D$ la matrice hessiana dimezzata di $\omega$. 
In un mezzo isotropico, $\nabla \cdot (D\nabla) = \nabla^2$. Inoltre, riscalando la variabile spaziale usando  $\mathbf{\zeta} = \mathbf{X} - T\mathbf{v}_g$, e $\tau = \epsilon T$, otteniamo
\begin{equation}
	i\partial_\tau +\nabla \cdot (D\nabla\psi) + \gamma |\psi|^2 \psi = 0
\end{equation}
che è proprio la NLSE in un mezzo anisotropo.
\end{document}
